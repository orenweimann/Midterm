\documentclass[12pt,a4paper]{article}
% these 3 make the hebrew work (ucs is from the culmus-latex package)
\usepackage{ucs}
\usepackage[utf8x]{inputenc}
\usepackage[english,hebrew]{babel}
% colors package with options to use color names in english
\usepackage[usenames,dvipsnames]{xcolor}
\usepackage{color}
% standard package for math symbols and equations
\usepackage{amsmath}
\usepackage{amssymb}
% package to change the page layout
\usepackage[margin=2cm]{geometry}
% package for footers and headers
\usepackage{fancyhdr}
%package for enumerate
\usepackage{enumitem}
%defining new list for questions
\newlist{questions}{enumerate}{1}
\setlist[questions]{label*=\textcolor{BurntOrange}{שאלה \arabic*: }}
% package for images
\usepackage{graphicx}
% paragraph indentation removal
\setlength{\parindent}{0pt}

%%%%%%%% macros %%%%%%%%
% course name, homework number, lecturer, 
% tutorer, hebrew date, due date
% definitions, theorem, examples, notes
% questions, solutions
\newcommand{\minititle}[1]{\par \bigskip \begingroup{\Large{\textbf{\underline{#1}}}}\endgroup \par \bigskip} 
\newcommand{\coursename}{מבני נתונים}
\newcommand{\semester}{סמסטר ב' תשע"ח}
\newcommand{\definition}{\textcolor{Violet}{\textit{הגדרה.}} }
\newcommand{\example}{\textcolor{OliveGreen}{\textit{דוגמא. }}}
\newcommand{\assistant}{נתן ולהיימר}
\newcommand{\lecturer}{אורן וימן}
\newenvironment{solution}{\textcolor{ForestGreen}{\textit{פיתרון:}} \\ \small}

\begin{document} 

% setting up footer and header
\pagestyle{fancy}
\fancyhead{}
\renewcommand{\headrulewidth}{0cm}
\fancyfoot{}
\fancyfoot[RE,RO]{\thepage}
% setting up the title
\begin{titlepage}
\begin{Large}
מספר תעודת זהות: \underline{\hspace{4cm}}
\end{Large}
\bigskip
\par
\begin{huge}
  \bf
  אוניברסיטת חיפה  \hfill
  החוג למדעי המחשב \par
  \semester \hfill 8102/5/3
  \par
  \begin{center}
  \coursename \\
  בחינת אמצע
  \end{center}
\end{huge}
\Large
\textbf{מרצה:} \lecturer
\\
\textbf{מתרגל:} \assistant
\par
\bigskip
\textbf{משך הבחינה:} שעה וחצי \\
\textbf{חומר עזר:} דף \L{A4} כתוב/מודפס משני צדדיו.
\par
\bigskip
\textbf{הנחיות:}
\begin{itemize}
\item עליכם לכתוב את התשובות על הטופס ולהגיש את כל הטופס ואת הטופס בלבד.
\item היציאה מהכיתה במהלך הבחינה אסורה.
\item קראו היטב כל שאלה וודאו שאתם מבינים אותה לפני שאתם מתחילים לענות עליה. אם יש שאלות, פנו למרצה.
\item כתבו בכתב יד ברור וקריא. ניתן לכתוב בעיפרון.
\end{itemize}
\textbf{מבנה הבחינה:}
\bigskip
\par
\begin{tabular}{ | c | c | c | }
\hline \R{נקודות} & \R{מספר סעיפים} & \R{שאלה} \\ \hline 
40 & 4 & 1 \\ \hline
60 & 3 & 2 \\ \hline
\end{tabular}
\vfill
\centering
\huge
\bf
בהצלחה !
\vfill
\end{titlepage} 
\newpage

\begin{Large}
\bf\underline{שאלה 1:} נכון / לא נכון \quad )כל סעיף 01 נקודות(
\end{Large}
\par
על כל אחת מהשאלות הקיפו נכון או לא נכון \underline{ונמקו את בחירתכם}. על הנימוק להיות קצר אך קולע.
\begin{itemize}
\item[א.] נכון / לא נכון\\
לכל שתי פונקציות חיוביות $f(n) , g(n)$: או ש $f(n) = O( g(n) )$ או ש $g(n) = O( f(n) )$.\\
\begin{solution}
 לא נכון. דוגמא נגדית: $f(n) = n$ ו $g(n)$ היא פונקציה המוגדרת
\L{
\[ 
g(n ) = \begin{cases} 
1 & \text{n even} \\
n^2 & \text{n odd}
\end{cases}
\]
}
\end{solution}
\item[ב.] נכון / לא נכון\\
בהינתן $m$ עצי \L{AVL} מגודל $k$ כל אחד, הנתונים ברשימה $T_1 \leftrightarrow T_2 \leftrightarrow \cdots \leftrightarrow T_m$ ואשר מקיימים $T_1 < T_2 \cdots < T_m$ )כל המפתחות ב $T_{i-1}$ קטנים מכל המפתחות ב $T_{i}$(, אפשר לבצע הכנה ב $O(m)$ זמן כך שניתן יהיה לחפש איבר ב $O(\log m + \log k)$ זמן. \\
\begin{solution}
נכון. הסבר: בתור הכנה נבנה מערך ממוין בגודל $m$ מהמפתחות של שורשי העצים. כדי לחפש מספר $x$, נחפש קודם עץ במערך לפי $x$, ואז נחפש את $x$ בשלושה עצים: העץ שהגענו אליו, הקודם לו והעוקב לו. אם $x$ נמצא במבנה אז הוא חייב להימצא באחד מהם. סיבוכיות: $O( \log m + \log k )$ שזה הזמן לחפש במערך ואז הזמן לחפש בעצים.
\end{solution}
\item[ג.] נכון / לא נכון\\
אם במימוש עץ \L{2-3} המתכנת עשה טעות וכעת הזמן לפיצול צומת בעץ לוקח $\Theta( \log n)$ זמן במקום $O(1)$ זמן )כאשר $n$ זה מספר האיברים הנוכחי בעץ(, אז הזמן לבצע סדרה של $m$ הכנסות מספרים שונים לעץ שמתחיל ריק, הוא כעת $\Omega( m \log^2 m )$.\\
\begin{solution}
לא נכון. הסבר: הזמן לבצע $m$ הכנסות בעץ \L{2-3} \textbf{רגיל} הוא $O( m \log m )$. מכיוון שמספר הפיצולים בכל הכנסה הוא $O(1)$ משוערך )ניקח כפוטנציאל את מספר הצמתים מדרגה $3$ כפול $\log n$(, אז יש $O(m)$ פיצולים בסך הכל. לכן שגיאת המימוש מוסיפה לכל היותר $O( m \log m )$ פעולות ולכן הזמן הוא עדיין $O( m \log m )$.
\end{solution}
\item[ד.]נכון / לא נכון\\
 בעץ מושרש עם $n$ צמתים שבו לכל צומת פנימי יש בדיוק $3$ בנים, אם סכום גדלי תתי העצים של כל שני אחים הוא לפחות כגודל תת העץ של האח השלישי, אז גובה העץ כולו הוא $O( \log n )$. \\
\begin{solution}
נכון. הוכחה: מכיוון שבין כל שלושה אחים יש אח עם $n/3$ צמתים לפחות אז האח השני בגודלו הוא לפחות בגודל $n/6$, כי סכום הגדלים של העצים הקטנים הוא לפחות הגודל של העץ הגדול. לכן בכל רמה בעץ, כמות הצמתים יורדת בפאקטור של $\frac{1}{6}$ לפחות ולכן הגובה הוא $O( \log n )$.
\end{solution}
\end{itemize}
\newpage

\begin{Large}
\bf \underline{שאלה 2:} תכנון מבנה נתונים \quad )כל סעיף 02 נקודות(
\end{Large}
\bigskip
\par למגדל פיקוח בשדה תעופה דרוש מבנה נתונים לבעיה הבאה. \\ בכל פעם שנכנס מטוס חדש לתחום של מגדל הפיקוח, מבנה הנתונים מקבל כקלט מהרדאר זוג $(I , Dist )$ שמתאים למטוס שנכנס, כאשר:
\begin{enumerate}
\item $I$ הוא מספר מזהה ייחודי של המטוס.
\item $Dist$ הוא המרחק במטרים של המטוס ממגדל הפיקוח )לא בהכרח ייחודי(.
\end{enumerate}
בכל פעם שיוצא מטוס מהתחום של מגדל הפיקוח, מבנה הנתונים מקבל כקלט מהרדאר את המספר המזהה $I$ של המטוס שיצא מהתחום.
\par
על מבנה הנתונים לתמוך בפעולות הבאות:
\begin{enumerate}
\item  הוסף מטוס חדש למבנה בעל מספר מזהה $I$ ומרחק $Dist$. 
\item מחק מטוס קיים בעל מספר מזהה $I$ מהמבנה.
\item בהינתן מספר מזהה $I$, מה המרחק של המטוס בעל מספר מזהה זה ממגדל הפיקוח. אם המטוס לא נמצא, הוצא הודעת שגיאה.
\item בהינתן שני מספרים ממשיים $r_1 < r_2$, מי הוא המטוס בעל המספר מזהה המקסימלי שנמצא במרחק לפחות $r_1$ ולכל היותר $r_2$ ממגדל הפיקוח. אם אין כזה, הוצא הודעת שגיאה.
\end{enumerate}
\begin{itemize}[leftmargin=*]
\item[א.] תכננו מבנה נתונים מתאים לבעיה זו. נתחו את סיבוכיות הזמן של הפעולות במונחים של: $n$ \L{-} מספר המטוסים השונים, $k$ \L{-} מספר המרחקים השונים.
\item[ב.] כיצד תשובתכם לסעיף א' תשתנה, אם ידוע שכל המספרים המזהים הם מספרים טבעיים בני $10$ ספרות? כיצד היא תשתנה אם ידוע שהמרחקים הם מספרים טבעיים בני $10$ ספרות?
\item[ג.] ידוע כי המרחק של מטוס הוא מספר טבעי בן $10$ ספרות, אך המספרים המזהים הם כלשהם )ממשיים(. \\
יש לתמוך בפעולה הבאה: בהינתן שני מספרי זיהוי $i<j$, הקטן ב $1$ את המרחק של כל המטוסים שמספר המזהה שלהם הוא בין $i$ ל $j$ )כולל(. אם המרחק של מטוס הופך להיות $0$, צריך למחוק אותו מהמבנה. \\
תכננו מחדש את המבנה ונתחו את סיבוכיות הזמן הגרוע ביותר והזמן המשוערך לכל אחת מהפעולות.
\end{itemize}
\newpage
\begin{center}סעיף א'\end{center}
\begin{solution}
\textbf{נתחזק:} עץ \L{AVL} שהמפתח בו הוא המרחקים הקיימים, עץ \L{AVL} בכל קודקוד שהמפתח בו הוא מספרים מזהים של מטוסים בעלי אותו מרחק ועץ \L{AVL} נוסף ששהמפתח בו הוא המספרים המזהים ובכל קודקוד רשום המרחק של המטוס.
בנוסף, בכל קודקוד נשמור שני שדות מקסימום: מספר מזהה מקסימלי של מטוסים בעלי אותו מרחק ומספר מזהה מקסימלי בתת העץ המושרש בקודקוד זה. נקרא להם מקסימום לוקאלי ומקסימום גלובאלי בהתאם.
\par
\textbf{ביצוע הפעולות:}
\begin{itemize}
\item הכנסה: מכניסים לעץ הראשון לפי המרחק ואז לעץ השני לפי המספר המזהה. מעדכנים שדות מקסימום בדרך חזרה לשורש לפי הכלל:
\[
global\_max(v) = \max \{ local\_max(v), global\_max(v.left), global\_max(v.right) \}
\]
בנוסף מכניסים את המטוס לעץ השלישי לפי המספר המזהה בלבד. סיבוכיות $O( \log n )$
\item הוצאה: אותו דבר רק שמוחקים במקום מכניסים. סיבוכיות $O( \log n )$
\item חיפוש: מחפשים לפי $I$ בעץ השלישי ומחזירים את המרחק שלו. סיבוכיות $O( \log n )$
\item שאילתת טווח: מחפשים את $r_1$ ואז את $r_2$ ולוקחים את המקסימום של כל המקסימומים הגלובליים בקודקודים שנמצאים בטווח $[r_1 , r_2 ]$. סיבוכיות $O( \log k )$
\end{itemize}
\end{solution}
\begin{center} סעיף ב' \end{center}
\begin{solution}
אם המספר המזהה הוא מספר טבעי בן $10$ ספרות אז נשתמש באותו מבנה, והסיבוכיות היא כעת $O(1)$ לכל הפעולות כי $k \leq n \leq 10^{10}$. אם המרחק הוא מספר טבעי בן $10$ ספרות, נשתמש גם באותו מבנה וכעת $k$ הוא קבוע אז הפעולה הרביעית לוקחת $O(1)$ ושאר הפעולות לוקחות $O( \log n )$.
\end{solution}
\begin{center} סעיף ג' \end{center}
\begin{solution}
נשתמש באותו מבנה, ואת הפעולה החדשה נבצע כך:\\
נחפש את $i$ ואת $j$ ואז ונעבור על כל המטוסים בטווח ונקטין להם את המרחק ב $1$. לכל מטוס שנמצא בטווח הזה נלך גם לעץ ששומר את המרחקים ונעדכן בו את הקודקודים המתאימים )נמחק מטוס במרחק $Dist$ ונוסיף מטוס במרחק $Dist-1$(.\\
סיבוכיות: במקרה הגרוע $O(n \log n )$ כי נצטרך לעדכן לכל המטוסים את המרחק, אבל מכיוון שאפשר להקטין לכל מטוס את המרחק מספר קבוע של פעמים, יוצא שהזמן המשוערך הוא $O( \log n )$. ניקח כפוטנציאל את:
\[
\phi = \left(\sum\limits_{i=1}^n dist(p_i) \right) \cdot \log n 
\]
כאשר $p_i$ הוא המטוס ה $i$ ו $dist$ זה המרחק שלו. בניתוח לפי פוטנציאל זה, אכן מקבלים שהזמן המשוערך הוא $O( \log n )$ )שזה רק הזמן הדרוש לחיפוש $i$ ו $j$(.
\end{solution}
\end{document}
